\documentclass[12pt]{article}
\usepackage[utf8]{inputenc}

\usepackage{float}
\usepackage{graphicx}

\usepackage[a4paper, margin=2.5cm]{geometry}
\usepackage{parskip}
\usepackage{enumitem}
\usepackage[charter]{mathdesign}  % comment out if you want Latin Modern again
\usepackage[dvipsnames]{xcolor}
\usepackage{subcaption}

\newcommand{\Hrule}{\noindent\rule{\textwidth}{0.4pt}}
\newcommand{\question}[1]{{\large \textcolor{BrickRed}{Question #1}}}

\title{Machine Learning assignment 4}
\author{TODO: name (student number)}
\date{< 30 April 2021}

\begin{document}

\maketitle

% Feel free to include relevant figures or 
% \begin{verbatim}
%   (pseudo)code
% \end{verbatim} 
% where there's no placeholder.

% There are extra comments about some questions in A4.ipynb.

\question{1: why can the two-dimensional data we stored in \texttt{X} be reconstructed so well using only one principal component?}



\question{2: why do neither the horizontal, nor the vertical reconstruction coordinates match the true coordinates exactly?}



\question{3: make a scatter plot that contains true and reconstructed values $(k=1)$, as well as the single principal component.}

\begin{figure}[h]
    \centering
    \includegraphics[width=.6\textwidth]{q3.pdf}
    \caption{TODO}
    \label{fig:q3}
\end{figure}

\question{4: can you come up with a feature transformation (for one or more column(s) of \texttt{X}) that would make the principal components express more of the data's variance?}



\question{5: what is a good number of principal components to continue with \textit{and why}? (Base your answer only on this training set.)}



\question{6: how do the computation time and accuracies differ between the repeated evaluations with and without PCA, and how do you explain this? (Specify $k$ if you do not use the above.)}



\question{7 (bonus): test your theory.}



\question{8: can you describe a situation where (or model for which) PCA would not help?}



\question{9: make scatterplots of 1: (train/test/all) points expressed in terms of the first two principal components; 2: the first two numerical features.}

\begin{figure}[h]
    \centering
    \begin{subfigure}{.49\textwidth}
        \includegraphics[width=\linewidth]{q9_1.pdf}
    \end{subfigure}
    \begin{subfigure}{.49\textwidth}
        \includegraphics[width=\linewidth]{q9_2.pdf}
    \end{subfigure}
    \caption{TODO}
    \label{fig:q9}
\end{figure}

\question{10 (2 points): what do the (first two) principal components represent when you think back to what the numerical features are based on?}



\end{document}
